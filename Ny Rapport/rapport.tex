%oneside/twoside for the printer
\documentclass[a4paper,12pt,oneside,final,swedish]{extarticle}
%article vs. extarticle
%Paket
\usepackage[latin1]{inputenc}	%�,�,�
\usepackage[T1]{fontenc}
\usepackage{graphicx}
\usepackage{times}			%typsnitt times kanske
\usepackage[swedish]{babel}
\usepackage[affil-it]{authblk}
\usepackage{geometry}	%lets you customize the page size

\geometry{
 margin=20mm
} 

\usepackage{fancyhdr} %include this to customize headers/footers
\usepackage{titling}

%-------------------------------------------------------------H�r b�rjar dokumentet 
\begin{document}
% Framsida

\author{Grupp 9\\\\Kalle Bladin\\Erik Broberg\\Emma Forsling Parborg\\Martin Gr�d}
\title{Simulering av Elastiska Material}
\clearpage\maketitle %genererar en separat titelsida
\thispagestyle{empty}
\date{2014-03-xx}
\affil{Link�pings Universitet}
\pagebreak
%\pagestyle{empty} %fancy, empty, plain, myheadings




%Sammanfattning
\begin{abstract}
\thispagestyle{empty}
H�r ska sammanfattningen skrivas in
\hfill
\end{abstract}
\pagebreak 

%Inneh�llsf�rteckning, Figurf�rteckning & Tabellf�rteckning
\tableofcontents  % chapter with the table of contents
\addtocontents{toc}{\protect\thispagestyle{empty}}
\listoffigures    % chapter with the list of figures
\addtocontents{lof}{\protect\thispagestyle{empty}}
\listoftables     % chapter with the list of tables
\addtocontents{lot}{\protect\thispagestyle{empty}}

%-----------------------------------------------------------
%Inledning
%\pagestyle{plain}
\pagebreak
\pagestyle{plain}
\setcounter{page}{1}
%Inledning
\section{Inledning}
Inledande text �r alltid lite trevligt

\subsection{Syfte}
Denna rapport syftar till att unders�ka hur verklighetstrogna simuleringar av elastiska material kan g�ras i tre dimensioner med hj�lp av mass-fj�der-d�mparsystem. Simuleringarna som demonstreras �r menade att kunna g�ras i realtid, och erbjuda en anv�ndare viss interaktion. 

%Bakgrund
\section{Bakgrund}
\subsection{Numeriska metoder}

%F�rstudier i MATLAB
\section{F�rstudier i MATLAB}
\subsection{Kraftekvation}

\subsection{Utveckling av MSD-system}

%Implementering
\section{Implementering}
\subsection{Grundl�ggande systemarkitektur}

\subsection{Indexeringsproblem}

\subsection{Implementering av kraftekvation}
Att implementera kraftekvationen f�r varje massa i ett MSD-system enligt Ekvation N, inneb�r att kraftvektorn till varje massa i, ber�knas. Med denna kraftvektor kan i sin tur accelerationen ber�knas f�r varje massa. I detta projekt utforskades tv� algoritmer f�r detta �ndam�l.
\subsubsection{Algoritm 1: baserad p� massorna}
\subsubsection{Algoritm 2: baserad p� kopplingarna mellan massorna}

\subsection{Implementering av numeriska integreringsmetoder}
\subsubsection{Eulers stegmetod}
\subsubsection{Runge Kutta}

\subsection{Stabilitets- och prestandatest}

%Resultat
\section{Resultat}
\subsection{Resultat fr�n f�rstudier}
\subsubsection{En dimension}
\subsubsection{Tv� dimensioner}

\subsection{Resultat fr�n implementering}
\subsection{Stabilitets- och prestandakrav}

%Beskrivning av systemet
\section{Beskrivning av systemet}
\subsection{Systemkrav}
\subsubsection{H�rdvara}
\subsubsection{Mjukvara}

\subsection{K�ra programmet}

%Diskussion
\section{Diskussion}

%Resultat
\section{Resultat}


%Referenser
\begin{thebibliography}{9}

\bibitem{pfleeger10softwareengineering}
Shari Lawrence Pfleeger och Joanne M. Atlee, \emph{Software Engineering, Fourth Edition, International Edition}, Pearson 2010
  
\end{thebibliography}

%Bilagor
\appendix

\section{Bilaga om ditten}
Lorem ipsum dolor sit amet, consectetur adipisicing elit, sed do eiusmod tempor incididunt ut labore et dolore magna aliqua. Ut enim ad minim veniam, quis nostrud exercitation ullamco laboris nisi ut aliquip ex ea commodo consequat. Duis aute irure dolor in reprehenderit in voluptate velit esse cillum dolore eu fugiat nulla pariatur. Excepteur sint occaecat cupidatat non proident, sunt in culpa qui officia deserunt mollit anim id est laborum.

\end{document}
